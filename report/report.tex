\documentclass[12pt]{article}

\usepackage{fullpage}
\usepackage{multicol, multirow}
\usepackage{tabularx}
\usepackage{standalone}
\usepackage{ulem}
\usepackage[utf8]{inputenc}
\usepackage[russian]{babel}
\usepackage{listings}
\usepackage[usenames,dvipsnames]{color}
\usepackage{indentfirst}
\usepackage{amsmath}
\usepackage{ragged2e}
\usepackage{tabularx}

\lstloadlanguages{C,[ANSI]C++,python}
\lstset{extendedchars=false,
	breaklines=true,
	breakatwhitespace=true,
	keepspaces = true,
	tabsize=4
}

\newcommand{\StudentName}{Данилычев Иван}
\newcommand{\Group}{8О-206М}
\newcommand{\CourseName}{Компьютерная графика}
\newcommand{\LabNum}{420}
\newcommand{\Subject}{Написание отчётов по лабораторным работам}

\renewcommand\tabularxcolumn[1]{>{\Centering}m{#1}}
\newcommand{\Descriptive}[2]
{
	\begin{tabularx}{\textwidth}{|X|}
	\hline
	\\[#1em]
	#2 \\
	\\[#1em]
	\hline
	\end{tabularx}
}

\begin{document}

\begin{flushright}
\Large{
	\CourseName \\
	Лабораторная работа №\,\LabNum \\
	<<\Subject>> \\
	\StudentName, \Group \\
}
\end{flushright}

\subsection*{Задание}
\Descriptive{5}{Текст задания и требования (берутся с GitHub)}

\subsection*{Теоретическая часть}
\Descriptive{10}{Примерно столько теоретического материала по теме лабораторки, желательно --- с формулами}

\subsection*{Скриншот(ы)}
\Descriptive{2}{Один-два скриншота работающей программы, где можно увидеть выполненное задание}

\subsection*{Практическая часть}
\Descriptive{15}{Фрагмент(ы) кода, \textbf{с помощью которого выполнено задание} (не надо вставлять всю лабу!): процесс отрисовки, расчёт освещения, код шейдера, $\ldots$}

\subsection*{Выводы}
\Descriptive{3}{
Мысли на тему сделанной лабы. \textbf{Плохо:} <<При выполнении данной лабораторной я научился X и Y. Были получены навыки Y и Z. $\ldots$>> \textbf{Хорошо:} описание того, зачем нужны только что изученные алгоритмы; обобщение изученного; подводные камни, за которые удалось зацепиться при выполнении и реализации
}
\end{document}
